\documentclass{article} %aprire tipo di documento
\title{Titolo}
\author{Autore}
\date{April 2023}

\begin{document} %apre il documento
  %commento
  \section{Titolo sezione}
    \subsection{Titolo sezione interne}
      \subsubsection{Titolo sezione internissima}
  \paragraph{Paragrafo} %non va a capo
  \noindent %per andare a capo senza indentare

  \textit{testo italico}
  \textbf{testo grassetto}
  \underline{testo sottolineato}
  \emph{testo enfatizzato}

  \begin{itemize} %inizia una lista puntata
  \begin{enumerate} %inizia una lista numerata
    %lista con numeri romani
    \usepackage{enumerate} %importa librerie
      \begin{enumerate}[I]
      \begin{enumerate}[i]
    %possibile creare sottoliste inserendo una lista al posto dell'item
  \end{itemize/enumerate}

  \begin{table}[h!]
    \centering %per centrarla
    \begin{tabular}{|c c c c c} %per definire la tabulazione
    \end{tabular}
      %| inserisce una riga verticale
      %c per centrare il testo, l per allinearlo a sinistra, r allinearlo a destra
      \hline linea orizzontale
      c1 & c2 & c3 & c4 %inserimento campi
      \\ %per andare a capo
      [0.5ex] %modifica la distanza tra le righe
    \caption{Descrizione tabella}
    \label{questaTabella} %utilizzabili anche per sezioni
    \ref{questaTabella}
  \end{table}

  \usepackage{graphicx}
  \begin{figure}[h!]
    \centering
    \includegraphics[scale=1]{nomeFileImmagine}
    \caption{descrizione}
  \end{figure}

  \begin{equation}
    \frac{x}{y}
    x^{y} %apice
    x_{i,j} %pedice
    \href{https://latexeditor.lagrida.com/} %altre formule
    $formula inline$
  \end{equation}

  \footnote{Nota a pié di pagina}
  \tableofcontents %crea la bibliografia (indice)
  \cite{...} %referenze, inserire il BibTeX in un file .bib
  \include{file} %includere un file .tex in un altro file

  %altre informazioni
  \href{https://www.overleaf.com/learn/latex/Free_online_introduction_to_LaTeX_(part_1)}
\end{document}
